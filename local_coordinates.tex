\documentclass[a4paper]{article}

\usepackage{amsthm,amsfonts,amsmath,amscd,amssymb,latexsym}
\usepackage{mathrsfs}
\usepackage{hyperref}

\newtheorem{PARA}{}[section]
\newtheorem{definition}[PARA]{Definition}
\newtheorem{theorem}[PARA]{Theorem}
\newtheorem{lemma}[PARA]{Lemma}


\newcommand{\B}{{\mathbb B}}
\newcommand{\C}{{\mathbb C}}
\newcommand{\D}{{\mathbb D}}
\newcommand{\F}{{\mathbb F}}
\renewcommand{\H}{{\mathbb H}}
\newcommand{\N}{{\mathbb N}}
\newcommand{\Q}{{\mathbb Q}}
\newcommand{\R}{{\mathbb R}}
\renewcommand{\S}{{\mathbb S}}
\newcommand{\T}{{\mathbb T}}
\newcommand{\Z}{{\mathbb Z}}
%
\newcommand{\cA}{{\mathcal A}} 
\newcommand{\cB}{{\mathcal B}}
\newcommand{\cC}{{\mathcal C}} 
\newcommand{\cD}{{\mathcal D}}
\newcommand{\cE}{{\mathcal E}}
\newcommand{\cF}{{\mathcal F}}
\newcommand{\cG}{{\mathcal G}} 
\newcommand{\cH}{{\mathcal H}}
\newcommand{\cI}{{\mathcal I}}
\newcommand{\cJ}{{\mathcal J}}
\newcommand{\cK}{{\mathcal K}}
\newcommand{\cL}{{\mathcal L}} 
\newcommand{\cM}{{\mathcal M}} 
\newcommand{\cN}{{\mathcal N}}
\newcommand{\cO}{{\mathcal O}}
\newcommand{\cP}{{\mathcal P}}
\newcommand{\cQ}{{\mathcal Q}}
\newcommand{\cR}{{\mathcal R}}
\newcommand{\cS}{{\mathcal S}}
\newcommand{\cT}{{\mathcal T}}
\newcommand{\cU}{{\mathcal U}}
\newcommand{\cV}{{\mathcal V}}
\newcommand{\cW}{{\mathcal W}}
\newcommand{\cX}{{\mathcal X}}
\newcommand{\cY}{{\mathcal Y}}
\newcommand{\cZ}{{\mathcal Z}}
%
\newcommand{\sA}{\mathscr{A}}    
\newcommand{\sB}{\mathscr{B}}    
\newcommand{\sC}{\mathscr{C}}    
\newcommand{\sD}{\mathscr{D}}    
\newcommand{\sE}{\mathscr{E}}    
\newcommand{\sF}{\mathscr{F}}    
\newcommand{\sG}{\mathscr{G}}    
\newcommand{\sH}{\mathscr{H}}    
\newcommand{\sI}{\mathscr{I}}    
\newcommand{\sJ}{\mathscr{J}}    
\newcommand{\sK}{\mathscr{K}}    
\newcommand{\sL}{\mathscr{L}}    
\newcommand{\sM}{\mathscr{M}}    
\newcommand{\sN}{\mathscr{N}}    
\newcommand{\sO}{\mathscr{O}}    
\newcommand{\sP}{\mathscr{P}}    
\newcommand{\sQ}{\mathscr{Q}}    
\newcommand{\sR}{\mathscr{R}}    
\newcommand{\sS}{\mathscr{S}}    
\newcommand{\sT}{\mathscr{T}}    
\newcommand{\sU}{\mathscr{U}}    
\newcommand{\sV}{\mathscr{V}}    
\newcommand{\sW}{\mathscr{W}}    
\newcommand{\sX}{\mathscr{X}}    
\newcommand{\sY}{\mathscr{Y}}    
\newcommand{\sZ}{\mathscr{Z}} 
%
\newcommand{\Om}{{\Omega}}
\newcommand{\om}{{\omega}}
\newcommand{\eps}{{\varepsilon}}
\renewcommand{\phi}{{\varphi}}
\newcommand{\ups}{{\upsilon}}
\newcommand{\ga}{{\gamma}}
\newcommand{\de}{{\delta}}
\newcommand{\De}{{\Delta}}
\newcommand{\La}{{\Lambda}}
\newcommand{\be}{{\beta}}
\newcommand{\al}{{\alpha}}
\newcommand{\la}{{\lambda}}
\newcommand{\Ga}{{\Gamma}}
%
\newcommand{\fhat}{{\widehat{f}}}
\newcommand{\uhat}{{\widehat{u}}}
\newcommand{\xhat}{{\widehat{x}}}
\newcommand{\Hhat}{{\widehat{H}}}
\newcommand{\Khat}{{\widehat{K}}}
\newcommand{\rhohat}{{\widehat{\rho}}}
\newcommand{\lahat}{{\widehat{\lambda}}}
\newcommand{\Thetahat}{{\widehat{\Theta}}}
%
\newcommand{\tSi}{{\widetilde{\Sigma}}}
\newcommand{\talpha}{{\widetilde{\alpha}}}
\newcommand{\tbeta}{{\widetilde{\beta}}}
\newcommand{\tga}{{\widetilde{\gamma}}}
\newcommand{\tmu}{{\widetilde{\mu}}}
\newcommand{\tnu}{{\widetilde{\nu}}}
\newcommand{\tom}{{\widetilde{\om}}}
\newcommand{\tLa}{{\widetilde{\Lambda}}}
\newcommand{\tA}{{\widetilde{A}}}
\newcommand{\tB}{{\widetilde{B}}}
\newcommand{\tF}{{\widetilde{F}}}
\newcommand{\tJ}{{\widetilde{J}}}
\newcommand{\tP}{{\widetilde{P}}}
\newcommand{\tU}{{\widetilde{U}}}
\newcommand{\tf}{{\widetilde{f}}}
\newcommand{\tg}{{\widetilde{g}}}
\renewcommand{\th}{{\widetilde{h}}}
\newcommand{\tu}{{\widetilde{u}}}
\newcommand{\tv}{{\widetilde{v}}}
\newcommand{\tw}{{\widetilde{w}}}
\newcommand{\tx}{{\widetilde{x}}}
\newcommand{\ty}{{\widetilde{y}}}
\newcommand{\tz}{{\widetilde{z}}}
\newcommand{\tcF}{{\widetilde{\mathcal{F}}}}
%
\newcommand{\coker}{{\mathrm{coker}}}   
\newcommand{\im}{{\mathrm{im}}} 
\newcommand{\range}{{\mathrm{range}}}   
\newcommand{\SPAN}{{\mathrm{span}}}   
\newcommand{\dom}{{\mathrm{dom}}}   
\newcommand{\Graph}{{\mathrm{graph}}} 
\newcommand{\DIV}{{\mathrm{div}}}   
\newcommand{\trace}{{\mathrm{trace}}} 
\newcommand{\sign}{{\mathrm{sign}}} 
\newcommand{\id}{{\mathrm{id}}} 
\newcommand{\Id}{{\mathrm{Id}}}
\newcommand{\rank}{{\mathrm{rank}}} 
\newcommand{\codim}{{\mathrm{codim}}}  
\newcommand{\diag}{{\mathrm{diag}}}  
\newcommand{\cl}{{\mathrm{cl}}} 
\newcommand{\dist}{{\mathrm{dist}}}  
\newcommand{\INT}{{\mathrm{int}}}  
\newcommand{\Vol}{{\mathrm{Vol}}}  
\newcommand{\supp}{{\mathrm{supp}}}  
\newcommand{\modulo}{{\mathrm{mod}}} 
\newcommand{\standard}{{\mathrm{std}}} 
\newcommand{\ndg}{{\mathrm{ndg}}} 
\newcommand{\INDEX}{{\mathrm{index}}}  
\newcommand{\IND}{{\mathrm{ind}}}  
\newcommand{\ind}{{\mathrm{Ind}}}
\newcommand{\grad}{{\mathrm{grad}}}  
\newcommand{\IM}{{\mathrm{Im}}} 
\newcommand{\RE}{{\mathrm{Re}}}  
\renewcommand{\Re}{{\mathrm{Re}}}  
\renewcommand{\Im}{{\mathrm{Im}}}  
%
\newcommand{\Lie}{{\mathrm{Lie}}} 
\newcommand{\Aut}{{\mathrm{Aut}}}   
\newcommand{\Out}{{\mathrm{Out}}} 
\newcommand{\Diff}{{\mathrm{Diff}}} 
\newcommand{\Vect}{{\mathrm{Vect}}}  
\newcommand{\Symp}{{\mathrm{Symp}}}   
\newcommand{\Ham}{{\mathrm{Ham}}}  
\newcommand{\Per}{{\mathrm{Per}}} 
\newcommand{\Rat}{{\mathrm{Rat}}}  
\newcommand{\Flux}{{\mathrm{Flux}}}  
\newcommand{\Map}{{\mathrm{Map}}}  
\newcommand{\Or}{{\mathrm{Or}}}  
\newcommand{\Res}{{\mathrm{Res}}}   
\newcommand{\Fix}{{\mathrm{Fix}}}  
\newcommand{\Crit}{{\mathrm{Crit}}}  
\newcommand{\Hom}{{\mathrm{Hom}}}   
\newcommand{\End}{{\mathrm{End}}}  
\newcommand{\Tor}{{\mathrm{Tor}}} 
\newcommand{\MOR}{{\mathrm{Mor}}}  
\newcommand{\OB}{{\mathrm{Ob}}}  
%
\newcommand{\VERT}{{\rm Vert}}   
\newcommand{\Hor}{{\rm Hor}}  
%
\newcommand{\cHZ}{{\mathrm{c}_{\mathrm{HZ}}}}  
\newcommand{\CZ}{{\mathrm{CZ}}}  
\newcommand{\FS}{{\mathrm{FS}}}
\newcommand{\w}{{\mathrm{w}}}
%%
\renewcommand{\i}{{\mathbf{i}}}
\renewcommand{\j}{{\mathbf{j}}}
\renewcommand{\k}{{\mathbf{k}}}
%
\newcommand{\ad}{{\rm ad}}
\newcommand{\Ad}{{\rm Ad}}
\newcommand{\point}{{\rm pt}}
\newcommand{\ev}{{\rm ev}}
\newcommand{\ex}{{\rm ex}}
\newcommand{\odd}{{\rm odd}}
\newcommand{\can}{{\rm can}}
\newcommand{\eff}{{\rm eff}}
\newcommand{\sym}{{\rm sym}}
\newcommand{\norm}{{\rm norm}}
\newcommand{\torsion}{{\rm torsion}}
\newcommand{\NORM}{{\rm norm}}
\newcommand{\MAX}{{\rm max}}
\newcommand{\dvol}{{\rm dvol}}
\newcommand{\loc}{{\rm loc}}
\newcommand{\const}{{\rm const}}
\newcommand{\hor}{{\mathrm{hor}}}
\newcommand{\symp}{{\mathrm{symp}}}
%
\newcommand{\comb}{{\mathrm{comb}}}
\newcommand{\Floer}{{\mathrm{Floer}}}
%
\newcommand{\G}{{\mathrm{G}}} 
\newcommand{\GL}{{\mathrm{GL}}} 
\renewcommand{\O}{{\mathrm{O}}} 
\newcommand{\SO}{{\mathrm{SO}}} 
\newcommand{\U}{{\mathrm{U}}} 
\newcommand{\SU}{{\mathrm{SU}}} 
\newcommand{\PU}{{\mathrm{PU}}} 
\newcommand{\SL}{{\mathrm{SL}}} 
\newcommand{\ASL}{{\mathrm{ASL}}} 
\newcommand{\PSL}{{\mathrm{PSL}}} 
\newcommand{\Sp}{{\mathrm{Sp}}} 
\newcommand{\fgl}{{\mathfrak{gl}}} 
\newcommand{\fo}{{\mathfrak{o}}} 
\newcommand{\fso}{{\mathfrak{so}}} 
\newcommand{\fu}{{\mathfrak{u}}} 
\newcommand{\fsu}{{\mathfrak{su}}} 
\newcommand{\fsl}{{\mathfrak{sl}}} 
\newcommand{\fsp}{{\mathfrak{sp}}} 
\newcommand{\fg}{{\mathfrak{g}}} 
%
\newcommand{\Cinf}{C^{\infty}}
\newcommand{\CP}{{\C\mathrm{P}}}
\newcommand{\CS}{{\mathcal{CS}}}
\newcommand{\CF}{{\mathrm{CF}}}
\newcommand{\HF}{{\mathrm{HF}}}
\newcommand{\BC}{{\mathrm{BC}}}
\newcommand{\YM}{{\cal{YM}}}
\newcommand{\Areg}{{\cal A}_{\rm reg}}
\newcommand{\Sreg}{{\cal S}_{\rm reg}}
\newcommand{\Jreg}{{\cal J}_{\rm reg}}
\newcommand{\JregK}{{\cal J}_{\rm reg,K}}
\newcommand{\Hreg}{{\cal H}_{\rm reg}}
\newcommand{\HJreg}{{\cal H}{\cal J}_{\rm reg}}
\newcommand{\XJreg}{{\cal X}{\cal J}_{\rm reg}}
\newcommand{\SP}{{\mathrm{SP}}}
\newcommand{\PD}{{\mathrm{PD}}}
\newcommand{\DR}{{\mathrm{DR}}}
\newcommand{\RP}{{\mathbb{R}\mathrm{P}}}
\newcommand{\excess}{{\mathrm{excess}}}
%
\newcommand{\inner}[2]{\langle #1, #2\rangle}
\newcommand{\INNER}[2]{\left\langle #1, #2\right\rangle}
\newcommand{\Inner}[2]{#1\cdot#2}
\newcommand{\winner}[2]{\langle #1{\wedge}#2\rangle}
%
\def\NABLA#1{{\mathop{\nabla\kern-.5ex\lower1ex\hbox{$#1$}}}}
\def\Nabla#1{\nabla\kern-.5ex{}_{#1}}
%
\def\abs#1{\mathopen|#1\mathclose|}
\def\Abs#1{\left|#1\right|}
\def\norm#1{\mathopen\|#1\mathclose\|}
\def\Norm#1{\left\|#1\right\|}
%
\renewcommand{\Tilde}{\widetilde}
\newcommand{\p}{{\partial}}
\newcommand{\notsub}{\not\subset}
\newcommand{\iI}{{I}}  
\newcommand{\bI}{{\partial I}} 
\newcommand{\multidots}{\makebox[2cm]{\dotfill}}
%
\newcommand{\IMP}{\Longrightarrow}
\newcommand{\IFF}{\Longleftrightarrow}
\newcommand{\INTO}{\hookrightarrow}
\newcommand{\TO}{\longrightarrow}
\newcommand{\longhookrightarrow}{\ensuremath{\lhook\joinrel\relbar\joinrel\rightarrow}}
\newcommand{\notimplies}{{\hspace{7pt}\not\hspace{-7pt}\implies}}
\newcommand{\qimplies}{{\hspace{18pt}?\hspace{-18pt}\implies}}

\title{Local coordinate expressions for the Donaldson geometric flow}
\author{Robin~S.~Krom}

\date{June 5, 2023}

\begin{document}

\maketitle

\begin{abstract}
This document contains the Donaldson geometric flow equations on the 4--dimensional torus expressed in local coordinates.
\end{abstract}

\section{Introduction}
The Donaldson geometric flow on the space of symplectic structures in a fixed
cohomology class $\sS_a$ is given by the evolution equation
\begin{equation}\label{eq:flow}
\p_t \rho = d*^\rho d\Theta^\rho, 
\end{equation}
where
\begin{gather*}
\Theta := *\frac{\rho}{u} -\frac{1}{2}\Abs{\frac{\rho}{u}}^2\rho, \qquad u:= \frac{\rho\wedge\rho}{2\dvol}.
\end{gather*}
If $M$ is a hyperK\"ahler surface, there exists symplectic forms $\om_1, \om_2, \om_3$ and complex structures $J_1, J_2, J_3$ such that each $J_i$ is compatible with $\om_i$ and the resulting metric
$$
\left<\cdot, \cdot \right> := \om_i(\cdot, J_i\cdot)
$$ 
is independent of $i$. In addition the complex structures satisfy the quaternion
relations $J_iJ_j = -J_jJ_i = J_k$ for every cyclic permuation $i,j,k$ of
$1,2,3$. In this case, the evolution equation becomes
$$
\p_t \rho = -d\sum_i dK_i^{\rho_t} \circ J_i^{\rho_t}, \qquad K_i^{\rho_t} := \frac{\om_i \wedge \rho_t}{\dvol_{\rho_t}}, \qquad \rho_t(J_i^{\rho_t}\cdot, \cdot) := \rho_t(\cdot, J_i\cdot).
$$
Now assume $M$ is the four-torus with the standard symplectic structures $\om_1, \om_2, \om_3$ and complex structures $J_1, J_2, J_3$. Let $x_0, x_1, x_2, x_4$ be the standard euclidian coordinates. Then, $dx_a \wedge dx_b$ for $a<b$ form an orthonormal basis for the two-forms $\Lambda^2(\R^4)$ and
$$
\rho = \sum_{a < b} \rho_{ab} dx_a \wedge dx_b.
$$
The evolution equation expressed in these local coordinates is given by
$$
\p_t \rho_{ab} = - \p_a\sum_i  \sum_c(A J_i A^{-1})_b^c \p_cK_i^\rho,
$$
where
\begin{gather*}
K_i^\rho := \frac{2(\rho_{0i} + \rho_{jk})}{u}, \quad i,j,k = 1,2,3\text{ cyclic}\\
u := \rho_{01}\rho_{23} + \rho_{02}\rho_{31} + \rho_{03}\rho_{12}\\
A:= \left[
\begin{array}{cccc}
0 & \rho_{0 1} & \rho_{0 2} & \rho_{0 3} \\
 - \rho_{0 1} & 0 & \rho_{1 2} &  - \rho_{3 1} \\
 - \rho_{0 2} &  - \rho_{1 2} & 0 & \rho_{2 3} \\
 - \rho_{0 3} & \rho_{3 1} &  - \rho_{2 3} & 0 \\
\end{array}
\right]\\
J_1 := \left[
\begin{array}{cccc}
0 & -1 & 0 & 0 \\
1 & 0 & 0 & 0 \\
0 & 0 & 0 & -1 \\
0 & 0 & 1 & 0 \\
\end{array}
\right], 
J_2 := \left[
\begin{array}{cccc}
0 & 0 & -1 & 0 \\
0 & 0 & 0 & 1 \\
1 & 0 & 0 & 0 \\
0 & -1 & 0 & 0 \\
\end{array}
\right],
J_3 := \left[
\begin{array}{cccc}
0 & 0 & 0 & -1 \\
0 & 0 & -1 & 0 \\
0 & 1 & 0 & 0 \\
1 & 0 & 0 & 0 \\
\end{array}
\right].
\end{gather*}
An alternative notation is
\begin{equation*}
\p_t \left[
\begin{array}{c}
\rho_{0 1} \\
\rho_{0 2} \\
\rho_{0 3} \\
\rho_{1 2} \\
\rho_{1 3} \\
\rho_{2 3} \\ 
\end{array}
\right] = - 
\left[\begin{array}{c}
\p_0 \\
\p_1 \\
\p_2 \\
\p_3
\end{array}
\right] \times \left(
\sum_i A \cdot J_i \cdot A^{-1} \cdot
\left[\begin{array}{c}
\p_0 K_i^\rho \\
\p_1 K_i^\rho \\
\p_2 K_i^\rho \\
\p_3 K_i^\rho
\end{array}
\right]\right).
\end{equation*}
The initial condition of the Donaldson flow is a symplectic structure $\rho_0$
in a fixed cohomology class. A symplectic structure is a 2-form $\rho$ that is
non-degenerate and closed, i.e.
$$
\rho\wedge \rho > 0, \qquad d\rho = 0.
$$
In local coordinates these two constraints read
$$
u > 0, \qquad \sum_{a < b < c} \partial_a \rho_{bc} = 0.
$$
Two symplectic forms have the same cohomology class, if and only if their total volume of the torus coincides.
$$
[\rho_1] = [\rho_2] = a \in \cH^2(M, \R) \Leftrightarrow \int_M \rho_1\wedge\rho_1 = \int_M \rho_2 \wedge \rho_2 = a^2(M). 
$$
Hence, given a symplectic structure $\rho_{ab}$ in local coordinates, we can compute
$$
\int_M u \dvol
$$
to obtain its cohomology class.

There is a two-dimensional analog to the Donaldson flow on closed surfaces. A detailed description of this flow and its connection to the Donaldson flow is given in \cite{SAL}. It is an evolution equation on the space of strictly positive functions given by
$$
\p_t u = d^*d \frac{1}{u}.
$$
Suppose the surface is the two-dimensional torus, then in the standard local coordinate system
$$
\p_t u = -\sum_a \p_a^2 \frac{1}{u}
$$
It can be shown with the help of a maximum-principle that, for a suitable
initial condition, a unique solution exists for all time and it converges to a
constant function.

\begin{thebibliography}{99}
\bibliographystyle{alpha}
\scriptsize
\vspace{-7pt}

\bibitem{DON1}
S.K.~Donaldson, 
Moment Maps and Diffeomorphisms.
{\it Asian J.\ Math.} {\bf 3} (1999), 1--16. \\
\url{http://bogomolov-lab.ru/G-sem/AJM-3-1-001-016.pdf}

\vspace{-7pt}

\bibitem{SAL}
D.A.~Salamon, Uniqueness of symplectic structures.
{\it Acta Mathematica Vietnamica} {\bf 38} (2013), 123--144. 
\url{http://www.math.ethz.ch/~salamon/PREPRINTS/unique.pdf}

\vspace{-7pt}

\bibitem{KROMSAL}
R.S.~Krom, D.A.~Salamon, The Donaldson geometric flow for symplectic
four-manifolds.
{\it Journal of Symplectic Geometry {\bf 17} (2019), 381--417}

\vspace{-7pt}

\bibitem{KROM}
R.S.~Krom, Regularity of the Donaldson Geometric Flow. {\it Acta Math Vietnam} {\bf 47},
611–633 (2022). \url{https://doi.org/10.1007/s40306-021-00454-x}
\end{thebibliography}
\end{document}
